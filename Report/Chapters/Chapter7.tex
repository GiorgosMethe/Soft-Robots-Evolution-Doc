% Chapter 7

\chapter{Conclusion} % Main chapter title

\label{Conclusion} % For referencing the chapter elsewhere, use \ref{Chapter1} 

\lhead{Chapter 7. \emph{Conclusion}} % This is for the header on each page - perhaps a shortened title

This thesis presented how virtual soft-robot morphologies can be evolved through novelty search, a diversity based search within neuroevolution of augmented topologies evolutionary algorithm and a generative encoding scheme, in VoxCad soft material simulator. All aspects of novelty search were investigated in detail including the affect behavior selection has in the evolution of the morphology and the locomotion strategy of soft-body structures, the performance when compared against traditional search method, such as fitness-based search, the obtained diversity in the phenotype level, and the role of sparsity  Additionally, different selection techniques were used in both novelty and fitness-based search, followed by a discussion on how these changes influences both search methods. A method of incorporating fitness information in novelty search was also introduced, resulting in a performance gain over pure novelty search method. Finally, both techniques were used to evolve soft-robots in four different gravity levels.


\todo{point out superiority of novelty search over all setting almost}\\

\todo{fitness based with competition better than pure}\\

\todo{point novelty search with fitness elitism better than pure}\\
