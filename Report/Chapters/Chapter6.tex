% Chapter 7

\chapter{Future Work} % Main chapter title

\label{Future Work} % For referencing the chapter elsewhere, use \ref{Chapter1} 

\lhead{Chapter 6. \emph{Future Work}} % This is for the header on each page - perhaps a shortened title

\section{Evolution of Materials}
Another aspect in the evolution of the morphology of soft robot bodies is the evolution of the materials alongside the structure. The scope of this thesis includes the evolution of soft robots morphologies given a set of predefined materials with their properties. The possibility of a dynamic palette of materials will enable more complex gaits for the soft bodies evolved. Using the same generative encoding, material properties can be added as output nodes in the genotype (CPPN), resulting in a palette of materials which size will be the same as the voxels presented in the evolved structure. Another possible way of evolving these properties could be that two genotypes can represent the same individual, following two different encoding. Direct encoding can be used for the material properties, whereas, mutations and crossovers will then only held among the same genotype types.