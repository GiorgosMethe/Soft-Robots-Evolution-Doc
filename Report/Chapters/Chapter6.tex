% Chapter 6

% Chapter 7

\chapter{Discussion} % Main chapter title

\label{Discussion} % For referencing the chapter elsewhere, use \ref{Chapter1} 

\lhead{Chapter 6. \emph{Discussion}} % This is for the header on each page - perhaps a shortened title

In this chapter the contributions of this thesis are discussed. The simultaneous evolution of morphology and locomotion strategy of soft-robots by novelty search has been investigated in depth. 

Neuroevolution of augmented topologies evolutionary algorithm used to evolve compositional pattern-producing networks, as the chromosome representation. CPPNs can be queried for the lattice dimension of VoxCad simulation software and output not only the morphology but also the distribution of materials in the structure of a soft-robot, determining the locomotion strategy that will be generated. The merits of using a generative encoding scheme like CPPNs were shown, where regular patterns in the output of these artificial networks exploited the properties of the problem task. This generative encoding showed its advantages in the case that random selection used in the evolution, whereas random generated soft-robots and direct-encoded soft-robots could not produce any decent locomotion strategy. 

For the first time in evolutionary soft-robotics a diversity based method such as novelty search was used, resulting in an unexpected and valuable scientific result. Novelty search method outperformed traditional fitness-based search in evolving soft-robots morphologies that can move fast in a virtual environment.
Both techniques used under the same objective function which was the displacement of soft-robots within a fixed time-span. All the experiments presented throughout this thesis proved that the search towards the above evaluation metric mentioned was not as successful as deploying novelty-search at the exact same settings, driving the search towards diversity in the behavior level. The resulted performance of novelty search method in this setting, showed that seeking for diversity in the behavior space can result in an improved performance in the fitness metric traditional methods optimize their population for. The affect that behavior selection has in the evolution of the morphology and the effectiveness of locomotion strategy of soft-body structures was investigated in detail. Several behavior metrics designed and deployed in novelty search, showing the performance difference when each one was used to define the behavior space. Moreover, it has been shown that a good behavior metric must contain information about the objective function which is subject to the optimization of the problem task. Previous work in evolving virtual creatures by novelty search~\citep{lehman2011evolving} used the resulted morphology of the robots created to determine the novelty of an individual. The resulted performance for pure novelty search method was worse than the fitness-based. Here, we verified that novelty search cannot perform equally good or better to fitness-based search when the objective function is not well defined. On the contrary, well defined behavior metrics can lead novelty search to outperform traditional fitness-based search in this settings. The obtained diversity in the phenotype level, and the role of sparsity were also investigated. Novelty search not only improved the performance and the diversity in the behavior space, but also contributed to the larger variety of virtual creatures evolved. Additionally, different selection techniques used in both novelty and fitness-based search, followed by a discussion on how this intermediate step in an evolutionary process can influence both search methods. 

The performance of fitness-based search was improved when competition was chosen as an intermediate step between generations. Competing individual in regards to their fitness value allowed the evolution to exploit the fitness landscape better than in the pure fitness method. In addition, competition was used in novelty search method, where competition in respect to global and local novelty between each species disturbed the properties of the novelty search method, resulting in worse performance than the pure novelty search. Competition within novelty search was applied in regards to the fitness of the individual solutions. Once again, this method was not as good as the pure method. Another method of incorporating fitness information in novelty search was also introduced, resulting in a significant performance gain over pure novelty search method. This was achieved via fitness elitism. Most fit individuals are passed through generations, carrying their valuable genes and giving the chance to the evolution to benefit from the mutations or the crossovers with these fit solutions. 

Finally, both techniques were used to evolve soft-robots in four variant gravity levels, showing interesting results and the possibility of influencing future robotic designs for planetary exploration. Experiments under these gravity levels verified that novelty search is indeed performing better than fitness-based search in the evolution towards high-velocity soft-robots. In addition some interesting characteristics of evolved locomotion strategies under variant gravity condition were also observed, adding knowledge and more possibilities when the gait of a soft-robot under low or high gravity is considered. With the progress three dimensional printing is showing, future space missions can benefit from low cost soft-robot explorers evolved to produce efficient locomotion. Passive locomotion can add more value to these soft-body structures, where environmental variable conditions can actuate certain material types to produce locomotion.

\section{Future Work} % Main chapter title

This section discusses possible future research that can be done in the topics approached and the contributions of this thesis. Designing soft-body structures in a simulated environment is a heavy task for designers~\citep{cheney2013unshackling}, evolutionary algorithms~\citep{stanley2002evolving} and generative encoding~\citep{stanley2007compositional} combined, succeeded in the evolution of these designs, as well as the coordination (distribution of materials) of the structures evolved. What follows in this chapter is points worth further investigation in the science of evolutionary soft-robotics, still in a simulated environment.

\subsection*{Evolution of materials}
The scope of this thesis included the evolution of soft robots morphologies given a set of predefined materials with specific properties. The reason behind this, is the fact that it is was only of interest to investigate ways of designing soft-robots having specific materials as building blocks. Another aspect in the evolution of the morphology of soft robot bodies is the evolution of the materials alongside the structure. The possibility of a dynamic palette of materials will enable more complex gaits for the soft-robots evolved. Using the same generative encoding, material properties can be added as output nodes in the genotype representation (CPPN), resulting in a palette of materials which size is the same as the voxels presented in the evolved structure. Another possible way of evolving these properties could be that two genotypes can represent the same individual, following two different encoding schemes. Direct encoding can be used for the material properties, whereas, mutations and crossovers will then only held among genotype of the same type.

\subsection*{Novelty search}
Incorporating fitness information into novelty search proved to be profitable for this diversity rewarding search, adopting somehow some of fitness-based search advantages. Fit individuals can be selected at the selection process resulting to their ability to survive throughout the generations and be able to optimize their chromosome further. Although, a method  that achieved in a significant gain over pure novelty search method was proposed, more ways of using genetic selection techniques can be used to achieve similar results.

As far as behaviors are concerned, a limited behavior space can benefit the search for novel individuals that their behavior belong only to this space, rewarding only those for their novelty. In this thesis, the space of behaviors was only normalized for trajectories of the robot bodies, whereas the orientation of their displacement played no role to the novelty search. A limited trajectory space could only take into consideration straight trajectories, treating all others as invalid behaviors and thus, not rewarding the individuals from which the trajectories were observed. It is expected that this type of novelty search will result in better solutions~\citep{lehman2011abandoning} as the diversity of locomotion patterns will only appeal to the strategy and not the direction. This technique used in~\citep{lehman2011abandoning}, called \emph{Minimal Criteria Novelty Search}, is a way of making the behavior space more compact so only ``good'' behavior will be rewarded for their novelty. Doing so, novelty search incorporates indirectly further fitness information. It would be very interesting to see how minimal criteria novelty search could perform in this setting. As an example of a minimal behavior space could be the set of straight trajectories.

\subsection*{Objective function in the evolution of soft-robot locomotion}

The objective function defined by previous work~\citep{cheney2013unshackling} and used throughout this thesis was the displacement of the soft-robot bodies within a limited time-span. Optimizing morphologies achieve the highest displacement possible is resulting to emerged morphologies that are able to move fast but are not stable enough to be considered valid. An objective function that could contain information about the stability of the virtual soft-robots would have been crucial to the evolution towards more stable gaits.