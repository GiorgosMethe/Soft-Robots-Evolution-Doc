% Chapter 7

\chapter{Future Work} % Main chapter title

\label{Future Work} % For referencing the chapter elsewhere, use \ref{Chapter1} 

\lhead{Chapter 6. \emph{Future Work}} % This is for the header on each page - perhaps a shortened title

This chapter discusses possible future research that can be done in the topics approached and the contributions of this thesis. Designing soft-body structures in a simulated environment is a heavy task for human designers~\cite{cheney2013unshackling}, evolutionary algorithms~\cite{stanley2002evolving} and generative encoding~\cite{stanley2007compositional} combined, succeeded in the evolution of these designs, as well as, and the coordination (distribution of materials) of the structures evolved. What follows in this chapter is points worth further investigation in the science of evolutionary soft-robotics, still in a simulated environment.

\paragraph*{Evolution of materials}~\\
The scope of this thesis includes the evolution of soft robots morphologies given a set of predefined materials with specific properties. The reason behind this, is the fact that is was only of interest to investigate ways of designing soft-robots having specific materials as building blocks. Another aspect in the evolution of the morphology of soft robot bodies is the evolution of the materials alongside the structure. The possibility of a dynamic palette of materials will enable more complex gaits for the soft bodies evolved. Using the same generative encoding, material properties can be added as output nodes in the genotype (CPPN), resulting in a palette of materials which size will be the same as the voxels presented in the evolved structure. Another possible way of evolving these properties could be that two genotypes can represent the same individual, following two different encoding. Direct encoding can be used for the material properties, whereas, mutations and crossovers will then only held among the same genotype types.

\paragraph*{Novelty search}~\\
Incorporating fitness information into the novelty search proved to be profitable for this diversity rewarding search adopting somehow some of fitness-based search advantages. Fit individuals can be selected at the selection process resulting their survival and optimization during the evolution, as long as there will not be new fitter individuals.

As far as behaviors are concerned, a limited behavior space can benefit the search for novel individuals that their behavior belong only to this space, rewarding only those for their novelty. In this thesis, the space of behaviors was only normalized for trajectories of the robot bodies, whereas the orientation of their displacement played no role to the novelty search. A limited trajectory space could only take into consideration straight trajectories, treating all others as invalid behaviors and thus, not rewarding the individuals from which the trajectories were generated. It is expected that this type of novelty search will result in better solutions~\citep{lehman2011abandoning} as the diversity of locomotion patterns will only appeal to the strategy and not the direction. This technique used in~\citep{lehman2011abandoning}, called \emph{Minimal Criteria Novelty Search}, is a way of making the behavior space more compact so only ``good'' behavior will be rewarded for their novelty. Doing so, novelty search then incorporates indirectly more fitness information, which was not a point of interest during this thesis.