% Chapter 1

\chapter{Introduction} % Main chapter title

\label{Chapter1} % For referencing the chapter elsewhere, use \ref{Chapter1} 

\lhead{Chapter 1. \emph{Introduction}} % This is for the header on each page - perhaps a shortened title

Soft robotics is a field of research inspired by soft-bodied organisms, whereas the engineering and designing aspects of soft-structures are in the center of interest. Soft-robotics can make the interaction between robots and living organisms safe. In addition, it allows soft robots to  function in more natural and  complex environments, where rigid robots have disadvantages. Actuated soft materials that react to environmental changes, add complexity to the designing phase of soft-robot engineering, since the infinite degrees of freedom of soft structures and the possible distributions of materials, make the number of possibilities vast.

Approaching such a deep search space, is a heavy task. Recent developments in evolutionary optimization though, have shown the possibility of successful evolution of both the morphology and the locomotion strategy for soft robotic structures, where the genotype representation is of vital importance to the evolution. Generative encoding has shown promising results especially in specific problem domains, such as evolving controllers for robot gait and morphology evolution. As direct encoding provides a direct mapping from genotype to phenotype level, indirect determines a set of rules, functions that can be queried and generate each individual in the space of the phenotype. Recent work has proved that evolutionary methods coupled with a generative encoding genotype representation can actually evolve both the morphology and the locomotion behavior of soft-robotics in a virtual simulation.

Traditional evolutionary methods in pursuance of the objective function, defined by the user, are blind to generate enough diversity within the population, often driving the evolution towards local optima. Novelty search, unlike traditional optimization methods does not aim to optimize individuals towards an objective, but instead, looks for novelty. Novelty search rewards diversity and leads to a boundless variety of solutions, mimicking natural evolution in such a way. Doing so, it has proven to be a successful method for searching vast spaces where the objective function is deceptive.

Soft-robotic structures have no limitations to the extent of possible morphologies that can be discovered. Within the same context, gravity conditions when robot locomotion strategies are investigated might be more decisive when it comes to the morphology of the robotic explorers. With the freedom soft-structures give to evolutionary techniques in respect to the designing part of the evolution (morphology), it is of interest to validate that a taxonomy of different locomotion strategies can be applied when the gravity acceleration varies. As these structures can be made completely out of soft materials, all constraints about the applicable shapes seize to exist, giving the opportunity to human or algorithms to design unconventional shapes.



\section{Thesis Contribution}

This thesis explores possibles ways of evolving the morphology and the locomotion strategy of soft structures under a virtual simulation environment. As baseline, an initial experiment is performed to confirm that these problems cannot be captured by a simple genetic method with direct encoding representation of genotype. Both direct and indirect encoding are also used under a random robot generator which shows the advantages of the latter in the produced structures, as well as, points out the need of a generative way to explore and exploit the geometrical properties of the problem. This encoding scheme is used paired to an evolutionary algorithm to verify results of previous work on the same domain, showing that generative representations for the genotype can indeed benefit these kind of evolutionary optimization methods. In addition, this thesis is exploring the effect of diversity based evolution can have in the performance of the evolved morphologies. Novelty search, a method rewarding the ``new'' in the behavior level is used for this purpose showing that not only same or better performance can be achieved through this method but also the diversity of behaviors is remarkably increased. Last, both search methods are used to evolve structures for a variety of gravity levels, expecting to show a different taxonomy of locomotion patterns under different conditions apart from general implications regarding the effect of gravity in the locomotion success of mobile machines.






\section{Thesis Outline}

Chapter~\ref{Background}, provides some background information on the field of soft robotics, an introduction to genetic algorithms, different encoding techniques for the genotype representation, neuroevolution algorithms, and finally, objective driven search is presented and compared to novelty search. In Chapter~\ref{Related Work}, related material about evolutionary techniques used to evolve artificial life, as well as the evolution of soft-robots morphology and locomotion are presented. Chapter~\ref{Method}, is a comprehensive documentation presenting details of the implementation of different evolutionary techniques. Chapter~\ref{Results}, gives a detailed presentation of the results achieved under different experimental setups. Next, in chapter~\ref{Future Work}, future applications and extensions of this work are provided. Chapter~\ref{Conclusion}, serves as an epilogue to this thesis, where the impact of the contributions are discussed.






