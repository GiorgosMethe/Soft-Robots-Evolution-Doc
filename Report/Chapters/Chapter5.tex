% Chapter 5

\chapter{Results} % Main chapter title

\label{Results} % For referencing the chapter elsewhere, use \ref{Chapter1} 

\lhead{Chapter 5. \emph{Results}} % This is for the header on each page - perhaps a shortened title

This chapter presents the results of the methods described earlier in this thesis. The performance of the evolutionary methods used for the co-evolution of the morphology and the locomotion strategy of soft-robots is discussed. In the previous chapter, it has been shown how random generated soft-robot morphologies fail to produce any locomotion abilities in this setting. Thus, the results obtained by these random methods are not discussed in this chapter. The performance of evolutionary methods is only presented. Pure novelty search is compared in respect to the goodness measure used in the simulations (displacement of soft-robots in body-lengths), against fitness-based search. The affect that novelty search has in the average and champion fitness of the population during the evolution is investigated in detail in the following sections. Additionally, both search methods are compared in respect to the number of novel behaviors they evolve during an evolution run. Moreover, the influence of the behavior metric in novelty search is shown, where all behavior metric proposed earlier are used to define the novelty of an individual. An altered number of closest behaviors in the sparsity equation of the novelty search, leads to interesting conclusion about the effect that has in the evolution process. Genetic selection techniques such as competition and elitism are also used to improve the baseline methods. More specifically, elitism is used in a proposed methodology to incorporate fitness information in novelty search. Last, the performance of both methods are investigated within variant gravity levels. Showing that gravity conditions do not have an effect in favor of a specific search method. Furthermore, a discussion of evolved locomotion strategies under different gravity conditions is showing how environmental condition can affect the evolved morphologies and strategies.

As in~\citep{cheney2013unshackling} and for comparison purposes, the population of each generation used is $30$, and the number of generations of the evolution is $1000$. For more details about evolutionary algorithm settings, see Appendix~\ref{EvolutionSettings}. For simulation settings used, see Appendix~\ref{SimulationSettings}. Due to computationally expensive simulations, not all experiments have been done using a lattice size of $10^3$, lattice sizes less than $10^3$ have been used as well, more specifically experiments have been done under $5^3, 7^3, 10^3$ lattice space, see Appendix~\ref{ExperimentalSettings}.



\section{Evolved Morphologies}

\begin{figure}[t!]
\centering
\begin{subfigure}[b]{1.0\textwidth}
\foreach \i in {1,2,3,4,5,6,7,8}{ 
\includegraphics[width=0.11\textwidth]{../Figures/Robots/fit-1-\i.jpg}
}
\caption{2-legged pull locomotion}
\label{fig:evolvedMorphologiesFitness-1}
\end{subfigure}
\begin{subfigure}[b]{1.0\textwidth}
\foreach \i in {1,2,3,4,5,6,7,8}{
\includegraphics[width=0.11\textwidth]{../Figures/Robots/fit-2-\i.jpg}
}
\caption{4-legged (nose \& tail) animal-like locomotion}
\label{fig:evolvedMorphologiesFitness-2}
\end{subfigure}
\begin{subfigure}[b]{1.0\textwidth}
\foreach \i in {1,2,3,4,5,6,7,8}{
\includegraphics[width=0.11\textwidth]{../Figures/Robots/fit-3-\i.jpg}
}
\caption{2-legged push-pull locomotion}
\label{fig:evolvedMorphologiesFitness-3}
\end{subfigure}
\begin{subfigure}[b]{1.0\textwidth}
\foreach \i in {1,2,3,4,5,6,7,8}{
\includegraphics[width=0.11\textwidth]{../Figures/Robots/fit-4-\i.jpg}	
}
\caption{2-legged galloping}
\label{fig:evolvedMorphologiesFitness-4}
\end{subfigure}
\caption{Champion (best overall) morphologies evolved in independent runs within fitness-based search. Each row illustrates the locomotion strategy of the individuals created. (Settings~\ref{Settings-size10})}
\label{fig:evolvedMorphologiesFitness}
\end{figure}

\begin{figure}[h!]
\centering
\begin{subfigure}[b]{1.0\textwidth}
\foreach \i in {1,2,3,4,5,6,7,8}{ 
\includegraphics[width=0.11\textwidth]{../Figures/Robots/nov-1-\i.jpg}
}
\caption{4-legged animal-like locomotion}
\label{fig:evolvedMorphologiesNovelty-1}
\end{subfigure}
\begin{subfigure}[b]{1.0\textwidth}
\foreach \i in {1,2,3,4,5,6,7,8}{
\includegraphics[width=0.11\textwidth]{../Figures/Robots/nov-2-\i.jpg}
}
\caption{2-legged galloping}
\label{fig:evolvedMorphologiesNovelty-2}
\end{subfigure}
\begin{subfigure}[b]{1.0\textwidth}
\foreach \i in {1,2,3,4,5,6,7,8}{
\includegraphics[width=0.11\textwidth]{../Figures/Robots/nov-3-\i.jpg}
}
\caption{L-shaped hopper}
\label{fig:evolvedMorphologiesNovelty-3}
\end{subfigure}
\begin{subfigure}[b]{1.0\textwidth}
\foreach \i in {1,2,3,4,5,6,7,8}{
\includegraphics[width=0.11\textwidth]{../Figures/Robots/nov-4-\i.jpg}
}
\caption{2-legged galloping}
\label{fig:evolvedMorphologiesNovelty-4}
\end{subfigure}
\caption{Champion morphologies evolved in independent runs within novelty search. Each row illustrates the locomotion strategy of the individuals created. (Settings~\ref{Settings-size10})}
\label{fig:evolvedMorphologiesNovelty}
\end{figure}


\begin{figure}[h!]
\centering
\begin{subfigure}[b]{1.0\textwidth}
\foreach \i in {1,2,3,4,5,6,7,8}{ 
\includegraphics[width=0.11\textwidth]{../Figures/Robots/fit-s5-1-\i.jpg}
}
\caption{Fitness based search}
\label{fig:evolvedMorphologies5-Fitness}
\end{subfigure}\\
\begin{subfigure}[b]{1.0\textwidth}
\foreach \i in {1,2,3,4,5,6,7,8}{ 
\includegraphics[width=0.11\textwidth]{../Figures/Robots/nov-s5-1-\i.jpg}
}\\
\foreach \i in {1,2,3,4,5,6,7,8}{ 
\includegraphics[width=0.11\textwidth]{../Figures/Robots/nov-s5-2-\i.jpg}
}
\caption{Novelty search}
\label{fig:evolvedMorphologies5-Novelty}
\end{subfigure}\\
\caption{Champion morphologies evolved in independent runs within fitness-based and novelty search in a lower resolution ($5^3$). Each row illustrates the locomotion strategy of the individuals created. (Settings~\ref{Settings-size5})}
\label{fig:evolvedMorphologies5}
\end{figure}


In this section some effective locomotion patterns evolved within fitness-based and novelty search will be discussed. Apart from the performance that the two methods achieved, both of them were successful in evolving effective strategies for the locomotion of the evolved morphologies. Figure~\ref{fig:evolvedMorphologiesFitness}, shows four different gait types evolved by fitness based search. All of these morphologies are considered ``good'' in respect to their fitness value, meaning that they achieve to travel up to $\sim 10$ body lengths during the simulation time ($0.4$ sec.). Considering that the lattice space used for this experiment was of size $10^3$. The produced low-resolution soft-robots cannot be compared with the real-life organisms. However, the results are shown that even in such low dimensions life-like locomotion can be evolved. 

For the fitness-based search, soft-body morphologies can use the front leg-s to pull themselves forward (see Fig.~\ref{fig:evolvedMorphologiesFitness-1}), evolve a four-leg locomotion where a nose and a tail are mostly used for stability (see Fig.~\ref{fig:evolvedMorphologiesFitness-2}), push and pull themselves forward (see Fig.~\ref{fig:evolvedMorphologiesFitness-3}), and gallop using both of their legs (see Fig.~\ref{fig:evolvedMorphologiesFitness-4}). 

Moving from fitness-based search to novelty search, locomotion strategies do not differ too much, since the resolution does not allow the virtual soft-robots to explore more locomotion techniques. However, novelty search proves its merits as far as the morphologies are concerned. More complicated structures are now evolved, which can be explained by the fact that, novelty search pushes the evolution to investigate new kinds of behaviors, resulting to more complicated topologies for the networks (CPPNs) representing the morphologies produced. Figure~\ref{fig:evolvedMorphologiesNovelty}, presents four champion morphologies and their locomotion strategies. Once again, two-legged galloping soft-robots (see Figs.~\ref{fig:evolvedMorphologiesNovelty-2}, \ref{fig:evolvedMorphologiesNovelty-4}), animal-like locomotion based on four legs (see Fig.~\ref{fig:evolvedMorphologiesNovelty-1}), and hopper soft-robots (see Fig.~\ref{fig:evolvedMorphologiesNovelty-3}) are evolved.

Having illustrated the types of locomotion patterns have been evolved in the specific resolution for the lattice ($10^3$), it is of interest to see what both search techniques can achieve in a lower resolution setting. Figure~\ref{fig:evolvedMorphologies5}, illustrates the results for both methods in a lower resolution setting ($5^3$). Both methods, achieve in evolving \emph{fit} soft-robots which can locomote efficiently. What is interesting though, is the fact that all experiments held by fitness-based search failed to produce the locomotion strategies evolved by novelty search. A ``quarter-pyramid'' shaped soft-robot (see Fig.~\ref{fig:evolvedMorphologies5-Fitness}) was the champion individual in almost all runs of fitness-based evolution in this setting, whereas novelty search came up with two-legged virtual creatures (see Fig.~\ref{fig:evolvedMorphologies5-Novelty}).

It has been shown how different locomotion strategies have been evolved under two different methods, using the same settings. The discussion following in the next sections is mostly focused on the performance comparison of novelty search against the traditional fitness-based method within CPPN-NEAT evolutionary method.





\clearpage


\begin{figure}[ht!]
\centering
\includegraphics[width=1.0\textwidth]{../Figures/Results/indRunnAvgSize7Fitness.pdf}
\caption{Best so far fitness, $10$ individual runs for fitness based search. (Settings~\ref{Settings-size7})}
\label{fig:indRunsAvgSize10Fitness}
\end{figure}
~
\begin{figure}[ht!]
\centering
\includegraphics[width=1.0\textwidth]{../Figures/Results/indRunnAvgSize7Novelty.pdf}
\caption{Best so far fitness, $10$ individual runs for novelty search. (Settings~\ref{Settings-size7})}
\label{fig:indRunnAvgSize10Novelty}
\end{figure}

\section{Into The Performance of Novelty Search}

Before comparing novelty search to fitness based search, it is of interest to show how they individually behave under the same simulation settings.

Figure~\ref{fig:indRunsAvgSize10Fitness} shows $10$ independent runs for fitness based search. Following the objective function's gradient fitness based evolution does small steps towards better and more optimized solutions from generation to generation. What is more, fitness based evolution often sticks into specific morphologies which then tries to optimize leading the evolution to stop at these local maxima.

Figure~\ref{fig:indRunnAvgSize10Novelty} shows $10$ independent runs for the novelty search under the same settings. 

When compared to the fitness based search (see Fig.~\ref{fig:indRunsAvgSize10Fitness}) a clear difference can be perceived. Evolving for novelty means that within the evolution only a novel behavior is rewarded instead of a good behavior or a behavior that leads to the optimization of the objective function. Big steps in the fitness value on all independent runs can be observed. Fit individuals in respect to the objective function for which novelty search has no information within the evolution process, are results of new novel behaviors that novelty search seeks for. 

Observing only big steps in the fitness, we can derive that there is no optimization of morphologies within novelty search. Initially, novel individuals are highly rewarded, these individuals can be very good in respect to the fitness or not. The novelty search algorithm does not take into account the goodness of these individuals. At the same time, it does not have any information regarding this goodness value either. On the next generation, mutations, crossovers, and copies of these novel individuals are not going to be highly variant in respect to their chromosome from their ancestors, resulting to similar behaviors. These not variant behaviors are not going to be remarkably rewarded in respect to their novelty value. Thus, highly novel individuals are producing less novel children in regard to their behavior in respect to the previously observed novel behaviors. These children solutions, even though their fitness can be higher than before, having the potential to be optimized further, will not have the chance to reproduce in the next generations and be improved, as their behaviors are not novel enough. 

\begin{figure}[t!]
\centering
\includegraphics[width=1.0\textwidth]{../Figures/Results/FitNovRandomDirectSize5.pdf}
\caption{Comparison of simple genetic algorithm (direct encoding) against \emph{random} - \emph{fitness} - \emph{novelty} search with generative encoding. Best so far fitness averaged over $10$ runs. (Settings~\ref{Settings-size5})}
\label{fig:FitNovRandomDirectSize5}
\end{figure}

\begin{figure}[t!]
\centering
\includegraphics[width=1.0\textwidth]{../Figures/Results/FitvsNovVsDirSize10.pdf}
\caption{Comparison of simple genetic algorithm (direct encoding) against \emph{fitness} - \emph{novelty} search with generative encoding. Best so far fitness averaged over $10$ runs. (Settings~\ref{Settings-size10})}
\label{fig:FitvsNovVsDirSize10}
\end{figure}

To extensively compare the performance achieved by novelty search method in the same experiment held under two different simulation settings (for sizes $5^3$ ,$10^3$), set side by side with fitness search, random search, and finally a simple genetic algorithm. Notice, that the first three methods are referring to a generative encoding (CPPNs) evolved by CPPN-NEAT evolutionary algorithm and using selection in respect to fitness, novelty and finally random selection, while the last uses a direct encoded genome driven by fitness. 

Two dimensional trajectories as described in the previous chapter (see~\ref{BehaviorNoveltySearch}), are used by novelty search in order to describe the novelty in the behavior space. The objective function that describes the goodness of solutions is the displacement of the soft-robot's center of mass from its initial position in body-lengths, and it is used for all fitness-based search methods. Random selection in CPPN-NEAT achieved choosing random selected individuals to breed on each generation. For direct encoding, direct encoded genomes represent the solutions as described in Section~\ref{DirectEncodingEvolution}.

Figure~\ref{fig:FitNovRandomDirectSize5}, presents the results for the low resolution soft-robots ($5^3$). The average best so far displacement of the soft-robots in body lengths is presented alongside the deviation error. Notice, the difference between novelty search and the other methods. Novelty evolves structures that are superior than any other method does in these settings. It should be mentioned that in such a small structures complex locomotion patterns cannot be evolved due to the stability issues of the simulator, and because of the fact that lightweight structures can be bouncy, leading to ball shaped structures capable of achieving large displacement from their initial positions. That being said, we still have to deal with an optimization problem, where local optima and global ones can be found as the number of the possible solutions in this setting, using 4 materials, is $\sim 2,3 \times 10^{87}$. Using the two-dimensional trajectories of the soft-robots, novelty search visits optimal solutions that none of the other methods does. Local optima can prevent fitness-based search to achieve the performance of novelty search. Encoding limitations in direct encoding cannot lead to optimal solutions for this settings. In the case of random search, the individuals of each generation are selected randomly to reproduce. Having neither the information about their fitness, nor the driving force of novelty search that seeks for novel behaviors, it fails to evolve any decent locomotion (there is no guarantee that random search will visit new behaviors). The only reason random search in CPPN-NEAT achieves to develop displacement of $\sim 5$ body-lengths, is the powerful encoding used (CPPNs). The simple genetic algorithm approach which uses a direct encoding to represent the structure of the soft-robots performs better than using random selection with an indirect encoding. Structural symmetry and regularity does not provide all of its merits in such a low resolution settings.

Moving to a higher resolution for the lattice, it is expected that generative encoding will prove its advantages over the direct encoding scheme~\citep{cheney2013unshackling,stanley2007compositional}. Furthermore, novelty search now has a more difficult task as the space of possible behaviors, two-dimensional trajectories, becomes larger as more complicated morphologies can now be produced (morphology space for $10^3$ lattice space: $9.3 \times 10^{698}$). These morphologies can achieve life-like locomotion. The same experiment as before held under a lattice resolution of $10^3$. Figure~\ref{fig:FitvsNovVsDirSize10}, presents the results of the four different methods in these higher resolution settings. Results reassure that novelty search achieves higher fitness on average against fitness-based search. Nevertheless, there is no tremendous difference as in the previous experiment. Both methods achieve to evolve the soft-robot structure with the highest fitness found in all experiments ($\sim 14$ Body lengths) so far. Novelty search behaves more constant in evolving individuals with high fitness in all runs, on the other hand most of individual runs of fitness search is being trapped in a low fitness local optima, trying to optimize specific individuals without trying to explore deeply the fitness landscape like novelty search does successfully. Random selection within CPPN-NEAT evolution produced low-fitness morphologies for soft-robots. The high difference between random selection evolution and novelty search proves that the behavior of novelty search cannot be considered as a random search. Backtracking (visiting same behaviors)of random search methods is a crucial variable of the evolution. The superiority of generative encoding (CPPN) over direct encoding can evidently be observed. Regular in shape morphologies can take advantage of their geometrical properties to locomote efficiently. CPPNs (see Section~\ref{CPPN}) are proven to be successful in creating these morphologies. The performance of direct encoding when a higher resolution lattice is used for the soft-robots, was radically decreased. The structure and morphology regularity is a necessity for soft-robots in order to perform decently in this resolution settings, a property that direct encoding cannot capture failing in this resolution.

\begin{figure}[t!]
\centering
\includegraphics[width=1.0\textwidth]{../Figures/Results/AvgGenerChampNoveltyFitnessSize7.pdf}
\caption{Fitness of the generation's champion (best individual) for \emph{fitness} - \emph{novelty} search averaged over $10$ runs. (Settings~\ref{Settings-size7})}
\label{fig:AvgGenerChampNoveltyFitnessSize7}
\end{figure}

\todo{here}

Another aspect of the evolution should be inspected is how the population of each generation is affected in respect to the best individual per generation, especially thinking about the these generation champions is the ones that result in the increased of novelty search when compared with fitness based search.

In figure~\ref{fig:AvgGenerChampNoveltyFitnessSize7}, the champions' fitness (Best fitness found within each generation) of each generation is plotted averaged over $10$ runs.

Recall, that novelty search does not have any information about fitness of individuals. In fitness based search there is a clear trend that champions of each generation are getting better through the evolution resulting to an approximately monotonically increasing function.

On the other hand, generations' champions in novelty search apart from the early improvement which is mainly caused by the generative encoding, follow a random pattern.

What it is interesting here to see is that even though that the solutions novelty search gives, in this settings (lattice size: $7^3$), are clearly better than the ones evolved by fitness based search, on average the champions during novelty search evolution are worse.

Hence, individuals that resulting in the increased performance of novelty search clearly lie on the tail of the fitness distribution on each generation.


\begin{figure}[t!]
\centering
\includegraphics[width=1.0\textwidth]{../Figures/Results/ViolinPlotsAvgGenFitSize7.pdf}
\caption{Distributions of average population fitness per generation over 10 runs for \emph{fitness}(Blue) - \emph{novelty} (Green) search with generative encoding. (Settings~\ref{Settings-size7})}
\label{fig:ViolinPlotsAvgGenFitSize7}
\end{figure}

In the same fashion, the average population fitness seems also affected by the different optimization methods. Figure~\ref{fig:ViolinPlotsAvgGenFitSize7} illustrates the distribution of population's average fitness over $10$ independent runs for \emph{novelty}-\emph{fitness} based search every $100$ generations. The resulted distributions which are shown in violin-like shapes clearly show that the average generation's fitness remains stable through the whole evolution ($1000$ generations) for both methods. What is more, the generation's average fitness is significantly lower for \emph{novelty} search, meaning that when the evolution is being carried towards novel behaviors there is no such guarantee that assumes novel new founds in the behavioral space will also be \emph{fit}. 

What we see in the last two figures, evidently shows that even though novelty search achieves in finding more ``fit'' solutions than fitness based search in the specific problem domain, the average fitness of both generation champions and population remain lower than in fitness based search.

\begin{figure}[t!]
\centering
\includegraphics[width=1.0\textwidth]{../Figures/Results/novelIndividualsFitNovComp.pdf}
\caption{Number of novel behaviors found up to generation number, averaged over 10 runs. The novelty measure is computed as the average distance from the $10$-nearest behaviors for \emph{fitness} - \emph{novelty} search with generative encoding. (Settings~\ref{Settings-size7})}
\label{fig:novelIndividualsFitNovComp}
\end{figure}

Until this point, the performance of both fitness and novelty search methods have been compared in the same objective metric such as the displacement of the produced soft body robots. The former method method tries to optimize genomes in respect to the specific objective function, while the latter moves its interest into creating diversity of the population in the behavioral space. As shown before, the novelty search achieves to create novel individuals which are not only novel in respect to how different behaviors they have from the rest of the population they exist into, but also they achieve higher average fitness than those they are optimized towards that objective. Inverting the objective function now such as our goal is to generate a wide variety of behaviors, in this case, two dimensional trajectories, we expect that a much larger set of novel behaviors will be created by novelty search. Figure~\ref{fig:novelIndividualsFitNovComp}, presents the number of unique behaviors the two evolutionary methods found, averaged over $10$ runs. The resulted graph shows that comparing these two methods is pointless as \emph{novelty} search can force the evolution towards spaces in the behavioral space that have not visited, finding more novel individuals, which does not happen in the fitness search. Surprisingly, \emph{novelty} achieves better performance than \emph{fitness} search in both objectives set so far, creating fit, and at the same time diverse solutions.

\begin{figure}[t!]
\centering
\includegraphics[width=0.49\textwidth]{../Figures/Behaviors/behaviorsNovelty.pdf}\	
\includegraphics[width=0.49\textwidth]{../Figures/Behaviors/behaviorsFitness.pdf}
\caption{Novelty search creates a vast amount of behaviors achieving in this way to find fit individuals, and avoid local optima of the solution space. (Settings~\ref{Settings-size7})}
\label{fig:behaviorSpaceDiversity}
\end{figure}

To visualize the difference in the behavior space of the two methods, figure~\ref{fig:behaviorSpaceDiversity}, illustrates all the stored found novel behaviors (two dimensional trajectories) found in one evolution run of novelty and fitness search using the same novel measure to determine the novelty of a behavior.




\subsection{How Behavior Selection Affects \emph{Novelty}-Search}

A good behavior metric should include information about the objective function. In case of locomotion gait of soft robots a trajectory can be highly informative as far as the displacement of the robot's body, as well as the gait, is concerned. Two robot bodies which travelled the same distance into an equal time horizon, should have the same fitness if displacement is only measured, nevertheless, the locomotion strategy, is something that can only affect the actual behavior metric and not the objective function. Forcing the evolution to seek for the novel in the behavior space results in producing $>10 \times$ more novel behaviors than \emph{fitness} search (depending on the threshold and the behavior metric), which  indirectly implies that high fitness individuals will be found as the behavior space is heavily searched. How the behavior metric affects the performance of the evolution is discussed in detail in the next section.

\begin{figure}
\centering
\includegraphics[width=1.0\textwidth]{../Figures/Results/FitNovSize5Pen2.pdf}
\caption[]{Best so far fitness averaged over $10$ runs, penalizing actuated materials\footnotemark for \emph{fitness} - \emph{novelty} search with generative encoding. (Settings~\ref{Settings-size5})}
\label{fig:FitNovSize5Pen2}
\end{figure}

\footnotetext{Actuated materials penalize fitness: \[f = (1 - (n_{actuated} / n_{total})^{1.5}) \times disp \], where $n_{actuated}$, is the number of actuated voxels, $n_{total}$ total number of voxels and $disp$ the displacement of the softbot's center of mass.}


The importance of selecting a good behavior metric is important in order for novelty search to explore the behavior space to a great extent. For example, searching for fast robots while you exploring the behavior space of their trajectories is a wise decision considering that all information needed to determine the fitness (speed) is incorporated inside the behavior (trajectories) assuming static sampling rate of the trajectories. In this experiment to investigate what is the result of the novelty-search evolution when no information about fitness is provided by the behavior, a objective function was selected that the currently used behavior metric doe not include information about. The two dimensional projection of the trajectories in $x,y$-axis are again selected, while instead of evaluating the fitness in displacement, this displacement is penalized by the number of actuated voxels are inside the structure of the soft robot. Figure~\ref{fig:FitNovSize5Pen2}, illustrates the best so far fitness for both novelty and fitness search averaged on $10$ independent runs. Comparing the results with figure~\ref{fig:FitNovRandomDirectSize5}, one can notice how novelty search performs poorly in this setting. Considering that the same method outperforms traditional fitness-search evolution when the whole information of the fitness function is contained in the behavior. Trying to find novel trajectories in the first case proved successful in respect to the final displacement of the individuals produced. On the other hand trying to maximize the distance and in the same time use as few actuated voxels as possible, proved crucial for the final outcome of the search for novelty. If the number of actuated voxels had been included in some way into the behavior metric, novelty-search would have been more exploratory towards this direction as well.


\begin{figure}[t!]
\centering
\includegraphics[width=1.0\textwidth]{../Figures/Results/BehaviorsPerformance.pdf}
\caption{Comparison of the evolution's best fitness result from $10$-runs under different behavioral metrics for \emph{novelty} search (right). \emph{Fitness} search is also evaluated under the same settings (left - \textcolor{MidnightBlue}{blue} box). (Settings~\ref{Settings-size7})}
\label{fig:BehaviorsPerformance}
\end{figure}


Choosing the appropriate metric to describe a phenotype into the behavior space is crucial in the performance of novelty search. Figure~\ref{fig:BehaviorsPerformance} illustrates the experimental results under different behavior metrics, alongside the performance of pure \emph{fitness} based optimization. A set of $10$ different behavior types was used including the three dimensional trajectories of the soft robots (3D-Traj), the two dimensional projection on $x,y$-axes of the previous behavior (2D-Traj), the pace sampled every $0.001$ sec. (Pace), the discrete Fourier transformation of the same signal which was sampled every $0.00001$ sec. (DFT-Pace), the voxels touching the ground on each time-step (VTG, DFT-VTG), the maximum pressure per time-step (Pr, DFT-Pr), and the kinetic energy of the whole structure (KE, DFT-KE). What is shown here, is the fitness in body lengths of the champion individual during the whole evolution from $10$-independent runs of the experiment. Both trajectory behavior types achieve the best performance as far as fitness is concerned, with a small difference in favor of the three dimensional one. Pressure is coming third achieving high performance close to the previous two trajectory behavior types, pace and kinetic energy of the structure are next in the performance ladder, and last one is the behavior signal that count how many voxels touch the ground on each sampling time-step. The results of using $10$ different behavior types can be clustered into three performance categories. The first one which includes the two types of trajectories and achieves the best performance of all, the second one which includes raw values and the discrete Fourier transformation of pace, pressure and kinetic energy, the last and worst one with the number of voxels touching the ground. 

\begin{figure}[t!]
\centering
\includegraphics[width=1.0\textwidth]{../Figures/Results/KnnExperimentSize5.pdf}
\caption{Best so far fitness averaged over $10$ runs, for different $k$ to sparsity computation of the behavior. (Settings~\ref{Settings-size5})}
\label{fig:KnnExperimentSize5}
\end{figure}

The performance of novelty search when trajectory of the soft bodies is used as a behavior metric is superior over all other behavior metrics. Trajectories are a very good selection for this kind of problem, since they can indirectly not only encode the objective function which is the displacement, but also the locomotion strategy and that is the reason why they explore better the landscape of behaviors resulting in such high difference in fitness against the fitness search. 

The rest of the behavior metrics apart from VTG and VTG-DFT, are close, as far as the final performance of the evolution is concerned. On reason that they fail to meet the trajectories performance is the fact that even though they keep track of features that can actually measure the performance of the robot, speed, they cannot encode the direction of the soft-body during the simulation. Which is crucial if we think that soft-robots having a circle trajectory, even though they can produce fast locomotion that displacement from their initial positions will remain low.

Counting the number of voxels in a structure that touch the ground in every timestep of the simulation, does not have any implication about how fast the robot is moving. A fast moving robot that is hopping can have the same behavior signature with a hopping robot that stays in the same position after each jump, yet, using the trajectories these two soft-robots will have a huge difference in their behaviors.

Within the same figure and on the left side of it (\textcolor{MidnightBlue}{blue} box), \emph{fitness}-search is also evaluated under the same experimental settings. The performance of this objective optimization method is only comparable with the worst \emph{novelty}-search scenario when the VTG behavior is selected for novelty to be measured.

\begin{figure}[t!]
\centering
\begin{subfigure}[b]{1.0\textwidth}
\includegraphics[width=0.19\textwidth]{../Figures/Robots/f_4_g_100.jpg}
\includegraphics[width=0.19\textwidth]{../Figures/Robots/f_4_g_200.jpg}
\includegraphics[width=0.19\textwidth]{../Figures/Robots/f_4_g_300.jpg}
\includegraphics[width=0.19\textwidth]{../Figures/Robots/f_4_g_400.jpg}
\includegraphics[width=0.19\textwidth]{../Figures/Robots/f_4_g_500.jpg}\\
\includegraphics[width=0.19\textwidth]{../Figures/Robots/f_4_g_600.jpg}
\includegraphics[width=0.19\textwidth]{../Figures/Robots/f_4_g_700.jpg}
\includegraphics[width=0.19\textwidth]{../Figures/Robots/f_4_g_800.jpg}
\includegraphics[width=0.19\textwidth]{../Figures/Robots/f_4_g_900.jpg}
\includegraphics[width=0.19\textwidth]{../Figures/Robots/f_4_g_1000.jpg}
\caption{Fitness based search}
\end{subfigure}\\
\begin{subfigure}[b]{1.0\textwidth}
\includegraphics[width=0.19\textwidth]{../Figures/Robots/n_4_g_100.jpg}
\includegraphics[width=0.19\textwidth]{../Figures/Robots/n_4_g_200.jpg}
\includegraphics[width=0.19\textwidth]{../Figures/Robots/n_4_g_300.jpg}
\includegraphics[width=0.19\textwidth]{../Figures/Robots/n_4_g_400.jpg}
\includegraphics[width=0.19\textwidth]{../Figures/Robots/n_4_g_500.jpg}\\
\includegraphics[width=0.19\textwidth]{../Figures/Robots/n_4_g_600.jpg}
\includegraphics[width=0.19\textwidth]{../Figures/Robots/n_4_g_700.jpg}
\includegraphics[width=0.19\textwidth]{../Figures/Robots/n_4_g_800.jpg}
\includegraphics[width=0.19\textwidth]{../Figures/Robots/n_4_g_900.jpg}
\includegraphics[width=0.19\textwidth]{../Figures/Robots/n_4_g_1000.jpg}
\caption{Novelty search}
\end{subfigure}
\caption{Fitness based search trying to optimize a specific structure while the search for novelty results in a variety of shapes. (Settings~\ref{Settings-size10})}
\label{fig:morphologies}
\end{figure}


\subsection{Sparsity in \emph{Novelty}-Search}

Sparsity (eq.~\ref{sparsenessEquation}) is a measure that defines if a newly found behavior is novel enough to enter the set of novel behaviors. Figure~\ref{fig:KnnExperimentSize5} presents the resulted best so far fitness given different values for $k \in \lbrace 1, 2, 5, 10, 20 \rbrace$. In principle $k$ can define how tolerate the algorithm can be with new behaviors. It is not certain that a specific value for $k$ should give the highest performance in fitness and it depends almost completely by the application. The only implication in choosing value for $k$ is that choosing large values should yield in a more detailed exploration in the behavior space, in the contrary using small values final set of behaviors will be denser in the behavior space. In the specific figure and experiment $k=10$ was the setting that led to the best performance.

\subsection{Diversity of Individuals in \emph{Novelty}-Search}

Figure~\ref{fig:morphologies}, shows the champions every hundred generations of a experimental run for novelty search and fitness based search. While the fitness based search is stuck trying to optimize a specific morphology of a soft robots, novelty search is taking a walk in behavior space unveiling new morphologies for the soft-body structures. The same motif appears in every independent run of fitness and novelty search, while novelty search achieves the a variety of morphologies fitness search is sticking to certain shapes, different in every run. It is obvious, that both search methods have their advantages and disadvantages. First, fitness-based search optimizes (optimized distribution of material within the structure) certain shapes during the evolution, at the same time novelty does not optimize them. Novelty search because it is a diversity based method, new shapes are evolved but not optimized. Next section discusses ways of combining both search methods merits.

\begin{figure}[b!]
\centering
\includegraphics[width=1.0\textwidth]{../Figures/Results/fitComp100_20percent.pdf}
\caption{Best so far fitness averaged over $10$ runs, with no competition, local competition in the complete population of each species for \emph{fitness} search. (Settings~\ref{Settings-size7})}
\label{fig:fitComp100_20percent}
\end{figure}

\section{How Selection Affects the Performance of Both Search Methods}

Discussed extensively in the previous section, selection is a process that picks individuals in order to breed, be mutated or copied into the next generation. It is the part of the evolutionary algorithm that is responsible for producing the new generation, based on the individuals which exist into the current one.


\subsubsection*{\emph{Fitness} Search}

Figure~\ref{fig:fitComp100_20percent}, presents the results for two different selection methods, random selection from the top $20\%$ (\textcolor{MidnightBlue}{Blue}) and competition within individuals from the complete current population (\textcolor{BrickRed}{Red}). As it was expected, competition, as well as, the fact that the whole population has the opportunity to breed, contribute to the diversity of the population. This can be easily seen in this figure, random selection within the top $20\%$ of the population does not allow solution to reproduce meaning that it does not explore weaker individuals, which can later after enough mutations become better than the potential of the rest of the population. The deviation of the first method gives a perfect clue about how narrow is the fitness landscape at the converged area of search when only the best of each generation are allowed to breed.

\subsubsection*{\emph{Novelty} Search}

Since the algorithmic framework is the same for both searches, competition can be applied in novelty search as well. Figure~\ref{fig:NoveltyCompetitionsSize5}, presents the results when competition is held among individual of the whole generation's population among species in respect to different metrics. Competition is held among individual regarding their novelty among the whole population of the evolution and the novelty value they obtain if they are only compared with their species population. In both cases the overall performance of the evolution averaged on $10$ runs is worse than the default setting in novelty search where individuals to breed are selected randomly from the top-$20\%$ of the population of each species. Both selection approaches are performing poorly set side by side with the default selection method. Selecting individuals with high novelty withing the species is crucial for the performance, since these individuals can have low novel value when compared with the global population, leading to steps backwards in the evolution towards highly novel individuals. On the other hand, when individuals are competing using their global novelty measure leads to a slightly better performance, still far from the default setting, meaning that highly novel individuals can actually produce more novel individuals when they are allowed to breed. In other words, competition disturbs  the properties of novelty search, while in fitness based search merits of selecting not only the fittest individuals were shown.


\begin{figure}[t!]
\centering
\includegraphics[width=1.0\textwidth]{../Figures/Results/NoveltyCompetitionsSize5.pdf}
\caption{Best so far fitness averaged over $10$ runs, for local competition held among the population of each species for \emph{novelty} search with generative encoding. (Settings~\ref{Settings-size5})}
\label{fig:NoveltyCompetitionsSize5}
\end{figure}



\section{Incorporate \emph{fitness} Information into \emph{Novelty}-Search}

The reason that novelty search is considered such an revolutionary search method is because it finds solutions for deceptive problems, where the fitness landscape is not a straightforward function. What makes it so unique, it is the fact that instead of looking for better solutions in respect to an objective function is looking for different solutions. In each generation of the novelty search there are some solutions that are very good regarding their objective function value, eventually these novel individuals will stop being selected for reproduction since their novelty metric value will be declined as more individuals with similar behaviors will be produced by mutations of the same novel individuals, and they won't be optimized as they could have been. Mutations and other genetic operations can optimize these fit individuals more, but this is something that happens in fitness-based search. These individuals (with high fitness value) can be seen as \emph{stepping~stones}~\citep{lehman2011abandoning} towards more optimized versions of them. Being blind to the objective function, novelty search will eventually stop producing new individuals out of them, which will lead to promising individuals being thrown out of the evolution process. 

Competition is a simple way of combining the two searches together, after each generation is produced competition is held over all the population within each specie, selecting individual, for reproduction, that have high value in the objective function. Figure~\ref{fig:NoveltyCompetitionsSize5}, illustrates the results of using the global fitness as a measure for selection among two generations. The results (\textcolor{ForestGreen}{Green} line), reveal that competition for fitness in a novelty search setting disturbs the balance of the evolution towards novelty, not allowing novelty search to expand the search in a greater extend, since it is not the case that selected fit individuals will lead in novel behaviors. 






\subsection*{Fitness Elitism in Novelty Search}

\begin{figure}[t!]
\centering
\includegraphics[width=1.0\textwidth]{../Figures/Results/CopyFitChampions10.pdf}
\caption{Best so far fitness averaged over $10$ runs, for \emph{novelty} search with and without copying \emph{fit} champions and \emph{fitness} search. (Settings~\ref{Settings-size10})}
\label{fig:CopyFitChampions10}
\end{figure}

It has been shown, how selecting individuals in respect to their fitness distracts the evolution in novelty search. Hence, a new method is proposed for incorporating fitness information into novelty search without perturbing with its pipeline. Elitism is the process of copying the best individual of each species into the next generation with a probability of mutating it first. In this way best individuals are preserved and can be optimized later, which considered to be a successful way of protecting the best of each species generation so they can contribute with their beneficial genes later in the evolution. Novelty search can include elitism in its selection process, and it does that by copying the most novel organisms of the current population of each species to the next. Since, there is no point of changing this function, elitism can be used also to copy fit individuals within novelty search method. 

The way these two elitism functions can be used depends on the population size, and the problem. Probabilistic methods can also be used combining both elitism functions. In the specific setting, both elitism function copy new individuals to the new generations. In this way the evolution towards novelty does not get disturbed, at the same time, highly fit individuals have the chance to be optimized further as long as they are the fittest within the species population.

Figure~\ref{fig:CopyFitChampions10}, illustrates the gain in performance when fitness elitism is used in novelty search method compared with pure novelty and fitness based search methods.
















\clearpage

\section{Evolving Soft-Robots for Outer Space}  

\begin{figure}[t!]
\centering
\includegraphics[width=0.8\textwidth]{../Figures/Results/GravityExperiment.pdf}
\caption{Novelty search performs equally good or better than fitness based search in all gravity conditions tested. (Settings~\ref{gravitySettings})}
\label{fig:gravityConditions}
\end{figure}

In this section, it is of interest to show how different environmental conditions can affect both the search and the type of locomotion produced by the evolved soft robots. Since, similar conditions of other planets in our solar system are difficult to be reproduced by the simulation environment is used, we only interested to replicate the previous experimental settings with variant gravity acceleration conditions. Fitness-based and novelty search are used again within the CPPN-NEAT evolutionary algorithm to evolve the morphology and the locomotion strategies of these voxel structures. For the novelty search two dimensional trajectories of the soft bodies are chosen as the behavior metric to evaluate the novelty of each. To keep the computational cost low, lattice dimension $7^3$ was used.

Default settings used in all previous experiments, was also used for Jupiter's and Earth's gravity accelerations, whereas the simulation time used for both was $0.4$ seconds. For Moon's and Mars' evolution runs, a higher temperature period was used ($0.050$ seconds), in order effective locomotion to take place, as higher frequencies tend not to allow soft-body structures produce any decent locomotion in lower gravity conditions. Furthermore, for the latter two gravity levels, the simulation time was larger, up to $1$ second for each evaluation, since the velocities generated by the soft-robots were significantly lower in such low gravity levels.

Figure~\ref{fig:gravityConditions}, illustrates the performance of these two search methods, in four different settings, each method for ten independent runs, the best fitness achieved by an individual averaged for all runs are shown, together with the deviation errors. Pure novelty search is compared with pure fitness-based search, and the results show that novelty search, while it tries to optimize the sparsity of newly generated behaviors (two-dimensional trajectories), it achieves in producing better or equally good locomotion for the soft-robots produced. The results on different settings verify that novelty search can indeed achieve higher performance in the specific problem domain. The rest of this section is discussing in detail findings when soft-robots were evolved in different planetary conditions. For visualization purposes all the following experiment have been set with a lattice dimension $10^3$.



\begin{figure}[t!]
\centering
\begin{subfigure}[b]{1.0\textwidth}
\foreach \i in {1,2,3,4,5,6,7}{ 
\includegraphics[width=0.132\textwidth]{../Figures/Robots/fit-g1-1-\i.jpg}
}
\caption{3-legged hopper (Fitness-Search)}
\label{fig:gravityRobots1.6-1}
\end{subfigure}
\begin{subfigure}[b]{1.0\textwidth}
\foreach \i in {1,2,3,4,5,6,7}{ 
\includegraphics[width=0.132\textwidth]{../Figures/Robots/fit-g1-2-\i.jpg}
}
\caption{1-legged hopper (Fitness-Search)}
\label{fig:gravityRobots1.6-2}
\end{subfigure}
\begin{subfigure}[b]{1.0\textwidth}
\foreach \i in {1,2,3,4,5,6,7}{ 
\includegraphics[width=0.132\textwidth]{../Figures/Robots/nov-g1-1-\i.jpg}
}
\caption{L-shaped hopper (Novelty-Search)}
\label{fig:gravityRobots1.6-3}
\end{subfigure}
\begin{subfigure}[b]{1.0\textwidth}
\foreach \i in {2,3,4,5,6,8,9}{ 
\includegraphics[width=0.132\textwidth]{../Figures/Robots/nov-g1-2-\i.jpg}
}
\caption{C-shaped hopper (Novelty-Search)}
\label{fig:gravityRobots1.6-4}
\end{subfigure}
\caption{\textbf{Moon}: Locomotion strategies evolved in low-gravity conditions (Moon) consist mostly of hopper soft-robots. (Settings~\ref{Settings-size10-moon})}
\label{fig:gravityRobots1.6}
\end{figure}

\subsection{Soft-Robots on Moon}

Locomotion strategies evolved under low gravity conditions, for the gravity of the Moon more specifically, showed that hopping behaviors can only produce effective locomotion in these conditions. Low gravity makes it difficult for the soft-body structures to grip on any surface and evolve something different than hopping. Figure~\ref{fig:gravityRobots1.6}, shows four different kind of hopper soft-robots on Lunar's gravity. However, the morphology of each hopper differs.









\begin{figure}[t!]
\centering
\begin{subfigure}[b]{1.0\textwidth}
\foreach \i in {1,3,5,6,7,8,10}{ 
\includegraphics[width=0.132\textwidth]{../Figures/Robots/fit-g3-1-\i.jpg}
}
\caption{2-legged galloping (Fitness-Search)}
\label{fig:gravityRobots3.7-1}
\end{subfigure}
\begin{subfigure}[b]{1.0\textwidth}
\foreach \i in {1,2,3,4,5,7,8}{ 
\includegraphics[width=0.132\textwidth]{../Figures/Robots/nov-g3-1-\i.jpg}
}
\caption{2-legged C-shaped hopper (Novelty-Search)}
\label{fig:gravityRobots3.7-2}
\end{subfigure}
\caption{\textbf{Mars}: Gravity acceleration on Mars allows both galloping and hopping locomotion strategies. (Settings~\ref{Settings-size10-mars})}
\label{fig:gravityRobots3.7}
\end{figure}

\subsection{Soft-Robots on Mars}

The locomotion effectiveness on Mars were higher when compared to this on Moon's gravity acceleration, making it possible for the virtual soft-robots to develop other kinds of gait using legs. Figure~\ref{fig:gravityRobots3.7}, presents two evolved creatures, where the one is galloping having a two-legged body (fig.~\ref{fig:gravityRobots3.7-1}), and the next is hopping having a C-shaped soft-body (fig.~\ref{fig:gravityRobots3.7-2}).






\begin{figure}[b!]
\centering
\begin{subfigure}[b]{1.0\textwidth}
\foreach \i in {1,2,3,4,5,6,7,8,9,10}{ 
\includegraphics[width=0.088\textwidth]{../Figures/Robots/fit-g9-1-\i.jpg}
}
\caption{2-legged galloping (Fitness-Search)}
\label{fig:gravityRobots9.8-1}
\end{subfigure}
\begin{subfigure}[b]{1.0\textwidth}
\foreach \i in {1,2,3,4,5,6,7,8,9,10}{ 
\includegraphics[width=0.088\textwidth]{../Figures/Robots/fit-g9-2-\i.jpg}
}
\caption{4-legged animal like locomotion (Fitness-Search)}
\label{fig:gravityRobots9.8-2}
\end{subfigure}
\begin{subfigure}[b]{1.0\textwidth}
\foreach \i in {1,2,3,4,5,7,8}{ 
\includegraphics[width=0.13\textwidth]{../Figures/Robots/nov-g9-1-\i.jpg}
}
\caption{Tumbleweed-like locomotion (Novelty-Search)}
\label{fig:gravityRobots9.8-3}
\end{subfigure}
\caption{\textbf{Earth}: Morphologies evolved in gravity conditions on Earth, show that life-like locomotion strategies can be generated by soft-body creatures in a simulated environment. (Settings~\ref{Settings-size10-earth})}
\label{fig:gravityRobots9.8}
\end{figure}

\subsection{Soft-Robots on Earth}

On higher gravity levels, life-like locomotion can be evolved. Figure~\ref{fig:gravityRobots9.8}, shows three different locomotion strategies evolved by fitness-based and novelty search on gravity conditions on Earth. Galloping from is again produced by evolved two-legged shaped body creatures (fig.~\ref{fig:gravityRobots9.8-1}). Interesting animal-like gait has also been evolved (fig.~\ref{fig:gravityRobots9.8-2}), verifying the connection there is between gravity and the evolved locomotion strategies of living organisms evolving on Earth for thousands of years. Tumbleweed-like locomotion (fig.~\ref{fig:gravityRobots9.8-3}) has also been generated under novelty search method, producing rolling soft-robots that can locomote efficiently. Fact that adds significance to the novelty-search method, since fitness-based search did not produce this kind of locomotion strategy. Tumbleweed is a concept of low-cost locomotion that has inspired robot designers for Mars' missions in the past~\citep{antol2003low}, and has been already deployed in Antarctica for testing purposes by NASA.









\begin{figure}[t!]
\centering
\begin{subfigure}[b]{1.0\textwidth}
\foreach \i in {1,2,3,4,5,7,8}{ 
\includegraphics[width=0.132\textwidth]{../Figures/Robots/fit-g2-1-\i.jpg}
}
\caption{2-legged walking (Fitness-Search)}
\label{fig:gravityRobots27.6-1}
\end{subfigure}
\begin{subfigure}[b]{1.0\textwidth}
\foreach \i in {1,2,3,4,5,7,8}{ 
\includegraphics[width=0.132\textwidth]{../Figures/Robots/fit-g2-2-\i.jpg}
}
\caption{Push-pull locomotion (Fitness-Search)}
\label{fig:gravityRobots27.6-2}
\end{subfigure}
\begin{subfigure}[b]{1.0\textwidth}
\foreach \i in {1,2,3,7,8,9,10}{ 
\includegraphics[width=0.132\textwidth]{../Figures/Robots/nov-g2-1-\i.jpg}
}
\caption{C-shaped hopper (Novelty-Search)}
\label{fig:gravityRobots27.6-3}
\end{subfigure}
\caption{\textbf{Jupiter}: Heavier structures on Jupiter's gravity level can locomote efficiently using several strategies. (Settings~\ref{Settings-size10-jupiter})}
\label{fig:gravityRobots27.6}
\end{figure}

\subsection{Soft-Robots on Jupiter}

Moving on to higher gravity levels, Jupiter, heavier structures can use galloping effective strategies for their locomotion. Figure~\ref{fig:gravityRobots27.6}, presents some of them when novelty and fitness-based search were used for the evolution of them. Galloping (fig.~\ref{fig:gravityRobots27.6-1}) is again considered to be an effective way of moving in such a high gravity, whereas thicker legs are evolved to withstand the heavy gravitational force. Push-pull worm-like locomotion (fig.~\ref{fig:gravityRobots27.6-2}), can also produce a decent velocity to the soft-robot. Finally hoppers are also evolved to this setting, while they are using more actuated materials.




Concluding, different locomotion strategies can be evolved on different gravity levels producing effective locomotion. Low gravities do not allow other kinds of locomotion apart from hoppers to be evolved, while higher gravity acceleration conditions allow more complicated behaviors to be evolved. In all settings, both search methods produced effective locomotion for the soft-body structures, however, the performance in regard to the objective measure defined, displacement of the body in body lengths, was higher for novelty search in almost all gravity settings.



