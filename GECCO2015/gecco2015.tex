\documentclass{sig-alternate}

\begin{document}

 \conferenceinfo{GECCO'15,} {July 11-15, 2015, Madrid, Spain.}
    \CopyrightYear{2015}
    \crdata{TBA}
    \clubpenalty=10000
    \widowpenalty = 10000

\title{Novelty Search of Soft Robot Morphologies for Space Exploration}
%\subtitle{[Extended Abstract]
%\titlenote{A full version of this paper is available as
%\textit{Author's Guide to Preparing ACM SIG Proceedings Using
%\LaTeX$2_\epsilon$\ and BibTeX} at
%\texttt{www.acm.org/eaddress.htm}}}

\numberofauthors{1}
\author{
\alignauthor
???
% \alignauthor
% Ben Trovato\titlenote{Dr.~Trovato insisted his name be first.}\\
%       \affaddr{Institute for Clarity in Documentation}\\
%       \affaddr{1932 Wallamaloo Lane}\\
%       \affaddr{Wallamaloo, New Zealand}\\
%       \email{trovato@corporation.com}
% % 2nd. author
% \alignauthor
% G.K.M. Tobin\titlenote{The secretary disavows
% any knowledge of this author's actions.}\\
%       \affaddr{Institute for Clarity in Documentation}\\
%       \affaddr{P.O. Box 1212}\\
%       \affaddr{Dublin, Ohio 43017-6221}\\
%       \email{webmaster@marysville-ohio.com}
% % 3rd. author
% \alignauthor Lars Th{\o}rv{\"a}ld\titlenote{This author is the
% one who did all the really hard work.}\\
%       \affaddr{The Th{\o}rv{\"a}ld Group}\\
%       \affaddr{1 Th{\o}rv{\"a}ld Circle}\\
%       \affaddr{Hekla, Iceland}\\
%       \email{larst@affiliation.org}
% \and  % use '\and' if you need 'another row' of author names
% % 4th. author
% \alignauthor Lawrence P. Leipuner\\
%       \affaddr{Brookhaven Laboratories}\\
%       \affaddr{Brookhaven National Lab}\\
%       \affaddr{P.O. Box 5000}\\
%       \email{lleipuner@researchlabs.org}
}

\maketitle
\begin{abstract}
don't worry about this, we gonna write it at the end
\end{abstract}

% A category with the (minimum) three required fields
\category{H.4}{Information Systems Applications}{Miscellaneous}
%A category including the fourth, optional field follows...
%\category{D.2.8}{Software Engineering}{Metrics}[complexity measures, performance measures]

\terms{Theory}

\keywords{soft robotics, novilty search, CPPN, HyperNEAT, VoxCAD}

\section{Introduction}
Motivation, space, small bodies, passive actuation, story of Rosetta/Philae - stupid rigid probe without locomotion, we can do better!

\section{Background}
\begin{itemize}
\item gaits at different gravity levels (Ariadna Space Gaits); fixed morphology, rigid body dynamics
\item soft robots
\item unshackling evolution paper
\end{itemize}

\section{Methodology}
\subsection{VoxCAD simulator}

\subsection{HyperNEAT + CPPN + Novelty}
behaviours

\subsection{Novelty + Fitness-based}

\subsection{Experimental setup}
parameters, gravity

\section{Results}
novelty (+ fitness) better than fitness
examples of cool creatures
taxonomy of the evolved creatures at different gravity levels (hoppers, 2-3-4 legs, crawler, tumbleweed) 

\section{Discussion}
what's the use of this?
getting inspiration for soft robotic probe / landers (tumbleweed)
come back to asteroid scenario (passime motion), would it have saved Philae?

\section{Conclusions}
our setup better than results from "unshackling evoltion"
methodology is suitable to design diverse gaits of soft robots at various gravity levels
future work: ensamble of behaviors, very low gravity environments and rotational parameters of small body linked to actuation frequency

\bibliographystyle{abbrv}
%\bibliography{sigproc}

\end{document}
